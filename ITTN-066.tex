\documentclass[PMO,authoryear,toc]{lsstdoc}
\usepackage[cache=false]{minted}
% !TeX TXS-program:compile = txs:///pdflatex/[-shell-escape]

\input{meta}

% Package imports go here.

% Local commands go here.

%If you want glossaries
%\input{aglossary.tex}
%\makeglossaries

\title{Kubernetes Policies}

% Optional subtitle
% \setDocSubtitle{A subtitle}

\author{%
Joshua Hoblitt
}

\setDocRef{ITTN-066}
\setDocUpstreamLocation{\url{https://github.com/lsst-it/ittn-066}}

\date{\vcsDate}

% Optional: name of the document's curator
% \setDocCurator{The Curator of this Document}

\setDocAbstract{%
Requirements and restrictions for services deployment upon Rubin Kubernetes clusters.
}

% Change history defined here.
% Order: oldest first.
% Fields: VERSION, DATE, DESCRIPTION, OWNER NAME.
% See LPM-51 for version number policy.
\setDocChangeRecord{%
  \addtohist{1}{YYYY-MM-DD}{Unreleased.}{Joshua Hoblitt}
}


\begin{document}

% Create the title page.
\maketitle

\section{Introduction}\label{sec:intro}

By default, Kubernetes allows pods to be configured to bypass a number of isolation abstractions.
These ``escape hatches`` are an important mechanism to allow core k8s functionality to itself be deployed on top of k8s.
However, in the interest of maintaining security proprieties, we need to restrict access to these features to the ``k8s platform team`` by policy.

In addition, to promote a stable system which provides consistent and predictable service, there is motivation to restrict resource usage such as requiring that all pods declare resource requests and limits.

\section{Terminology}\label{sec:term}

The key words "MUST", "MUST NOT", "REQUIRED", "SHALL", "SHALL NOT", "SHOULD", "SHOULD NOT", "RECOMMENDED", "NOT RECOMMENDED", "MAY", and "OPTIONAL" in this document are to be interpreted as described in BCP 14 [RFC2119] [RFC8174] when, and only when, they appear in all capitals, as shown here.

\section{Pods}\label{sec:pod}

\subsection{Resource \code{requests} and \code{limits}}

All containers within a pod SHALL declare resource requests and limits for memory and cpu.

XXX does this requirement include initContainers?

\subsubsection{Example \code{podSpec} fragment}

\begin{minted}{yaml}
    resources:
      limits:
        cpu: "2"
        memory: 12Gi
      requests:
        cpu: "1"
        memory: 8Gi
\end{minted}

Containers should set a fairly conservative limits to prevent suddent load spikes impacting other pods schedule upon the same node.

\subsubsection{Maxmimum Allowed Limits}

\begin{center}
\begin{tabular}{|l|l|}
\hline
    \bf cpu & \bf memory \\ \hline
    8 & 32Gi \\ \hline
\end{tabular}
\end{center}

\subsection{Priviledged Containers}\label{sec:priviledged}

Privileged \code{containters} and \code{initContainers} SHALL be used only by the infrastructure team to provide platform level services.

\subsubsection{Example podSpec fragment}

\begin{minted}{yaml}
    securityContext:
      privileged: false
\end{minted}

\subsubsection{Discussion}

Common community provided deployments have been identified which use \code{priviledged initContainers} to modify non-namespaced sysctls which will affect all containers run upon the node and have the potential to make the system unstable.

Example:

\begin{minted}{yaml}
      initContainers:
      - command:
        - sysctl
        - -w
        - vm.max_map_count=262144
        securityContext:
          privileged: true
\end{minted}

\subsection{Pod Security Policies}

\code{PodSecurityPolicy} was removed in k8s 1.25 and SHALL NOT be used.

\subsubsection{Discussion}

The usage of \code{PodSecurityPolicy} would prevent routine updates to k8s >= 1.25.

\subsection{\code{hostPath} Volumes}\label{sec:hostpath}

\code{hostPath} volumes SHALL NOT be used.

\subsubsection{Discussion}

\code{hostPath} allows access to filesystems mounted on the host. This potentially allows the container to access secrets such as \code{/etc/shadow}, escape the container via \code{/var/run/docker.sock}, modify kernel parameters via \code{/proc}, \code{/sys}, \code{/dev}, and/or edit configuration files.

\subsection{Linux/Host Namespaces}\label{sec:hostns}

Kubernetes allows pods to ``opt-out`` of various Linux kernel namespaces which are the basis of container isolation.

Pods SHALL NOT set any of these \code{podSpec} keys to \code{true}:

\begin{itemize}
  \item \code{hostIPC}
  \item \code{hostNetwork}
  \item \code{hostPID}
\end{itemize}

Pods SHALL NOT set any of these \code{podSpec} keys to \code{false}:

\begin{itemize}
  \item \code{hostUsers}
\end{itemize}

Pods MAY set \code{hostUsers} to \code{true} but it is NOT RECOMMENDED.

\subsubsection{Example of forbiden podSpec fragment}

\begin{minted}{yaml}
  hostIPC: true
  hostNetwork: true
  hostPID: true
  hostUsers: false
\end{minted}

\subsubsection{Example of allowed but discouraged podSpec fragment}

\begin{minted}{yaml}
  hostUsers: true
\end{minted}

\subsubsection{Discussion}

As of k8s 1.25, \code{true} is the default value.
If k8s \code{userns} support reaches ``GA`` in a future release, it is probable that setting \code{hostUsers} to \code{true} will become forbidden.

\section{Services}\label{sec:svc}

\subsection{Service Types}

The following service types SHALL be allowed:

\begin{itemize}
  \item \code{ClusterIP}
  \item \code{ExternalName}
  \item \code{LoadBalancer}
\end{itemize}

The \code{NodePort} service type SHALL NOT be used.

\subsubsection{Discussion}

The \code{NodePort} service type bypasses the CNI overlay and the pool of external IPs allocated for use by \code{LoadBalancer} services.

\section{Namespaces}\label{sec:ns}

\subsection{\code{default} namespace}

The \code{default} namespace SHALL NOT be used.

\subsubsection{Discussion}

Resources being created within the \code{default} namespace is often an indication that a namespace was not properly specified for a deployment. It is also preferable to keep services contained within their own \code{ns} rather than mixed together in a common \code{ns}.

\section{Exceptions}\label{sec:except}

\subsection{DDS Pods}

Pods which communicate over \href{https://en.wikipedia.org/wiki/Data_Distribution_Service}{Data Distribution Service (DDS)} have a limited exception from these policies:

\begin{itemize}
  \item \ref{sec:hostpath}
  \item \ref{sec:hostns}
  \item \ref{sec:priviledged}
\end{itemize}

\subsubsection{Discussion}

DDS usage at the summit is in the process of being replaced with Kafka.
The migration is envisioned to be complete before the end of CY23.
As the remaining life time for DDS is limited, and it is planned to be retired prior to the start of Rubin Operations, it has been decided to is acceptable to exempt the existing deployments from new security policies.
The DDS deployments do not meet the guidelines of this document MUST be retired in order for the summit to be ready for an external security audit.

\section{Job and CronJob}\label{sec:job}

\appendix
% Include all the relevant bib files.
% https://lsst-texmf.lsst.io/lsstdoc.html#bibliographies
\section{References} \label{sec:bib}
\renewcommand{\refname}{} % Suppress default Bibliography section
\bibliography{local,lsst,lsst-dm,refs_ads,refs,books}

% Make sure lsst-texmf/bin/generateAcronyms.py is in your path
\section{Acronyms} \label{sec:acronyms}
\addtocounter{table}{-1}
\begin{longtable}{p{0.145\textwidth}p{0.8\textwidth}}\hline
\textbf{Acronym} & \textbf{Description}  \\\hline

DM & Data Management \\\hline
\end{longtable}

% If you want glossary uncomment below -- comment out the two lines above
%\printglossaries





\end{document}
